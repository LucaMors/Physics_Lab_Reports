\section*{Introduction}
When a positron is emitted inside a material, it can bind with an electron forming the so called Positronium. 

Positronium has two short lifetime bound states: the singlet para-Positronium~(p-Ps) and the triplet orto-Positronium~(o-Ps). They decay with the emission of two and three photons respectively, with a  life time 0.125~ns for the former and 142~ns for the latter. 
The aim of this report is to describe the experiment performed to study Positronium. The experiment has two main purposes:
\begin{itemize}
 \item Evaluate the decay ratio between o-Ps and p-Ps.
 \item Measure the temporal distribution for the two and three photons decays.
\end{itemize} 


\section*{Experimental Set Up}

To study the Compton scattering three inorganic NaI(Tl) scintillators with 7.5~cm diameter and height were used~(see Fig.~\ref{Fig:Set_up}). A $^{22}$Na $\gamma$-source was positioned inside a collimator lead brick between the Tagger and Scatterer detectors, while the scattered photon Detector was placed on a rotating arm in order to select the desired scattering angle.

The last dinode outputs of the PMT were sent to a Quad Linear Gate FAN-IN/OUT mod. Philips 744 in order to split them. One output then was sent directly to a CAEN digitizer mod. DT5720, an ADC with a sampling rate of 250 Ms/s and a resolution of 12 bit, while the other ones were sent to a  CFTD module. The timing signals were needed to produce the coincidences between the three detectors through a 32 inputs Logic Units mod. LU 278. These signals were used as trigger input for the ADC unit.
