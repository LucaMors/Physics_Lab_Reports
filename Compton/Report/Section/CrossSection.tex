\newpage
\section*{Differential Cross Section of Compton Scattering}
The differential cross section of the Compton scattering was derived by Klein and Nishina in 1929 and its expression is:
\begin{equation*}
	\frac{d\sigma}{d\Omega}(\theta)=\frac{r_e ^2}{2}\left(\frac{h \nu_f}{h \nu_i}\right)^2\left(\frac{h \nu_f}{h \nu_i}+\frac{h \nu_i}{h \nu_f}-sen^2(\theta)\right)
\end{equation*}


In order to experimentally measure the differential cross section and make a comparison with the theoretical value, the Detector was rotated at 90$^\circ$ and the Scatterer was replaced by an aluminum sample of  diameter $\phi$ 3.4~cm and thickness~h 0.7~cm.  

Initially  a first data acquisition was performed without the aluminum scatterer in order to have a background spectrum for the two detectors. Then  