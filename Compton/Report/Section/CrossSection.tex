\newpage
\section*{Differential Cross Section of Compton Scattering}
The differential cross section of the Compton scattering was derived by Klein and Nishina in 1929 and its expression is:
\begin{equation*}
	\frac{d\sigma}{d\Omega}(\theta)=\frac{r_e ^2}{2}\left(\frac{h \nu_f}{h \nu_i}\right)^2\left(\frac{h \nu_f}{h \nu_i}+\frac{h \nu_i}{h \nu_f}-sen^2(\theta)\right)
\end{equation*}


In order to experimentally measure the differential cross section and make a comparison with the theoretical value, the Detector was rotated at 90$^\circ$ and the Scatterer was replaced by an aluminum sample of  diameter $\phi$ 3.4~cm and thickness~h 0.7~cm.  

Initially  a first data acquisition was performed without the aluminum scatterer to store a background spectrum for the two detectors, then the real acquisition was done using the Al sample. For both the session the coincidences between the detectors were used as trigger signals for the digitizer, moreover the total number of events detected by the Tagger was recorded using a CAEN scaler. Fig.~ presents the acquired spectra with and without the aluminum sample, while Fig.~ 

The experimental cross section was then calculated as:
\begin{equation*}
	\left[\frac{d\sigma}{d\Omega}(\theta)\right]_{exp}=\frac{\Sigma_\gamma}{\varepsilon N \Delta\Omega I/s}
\end{equation*}
 where $\Sigma_\gamma$ is the number of events in the full energy peak in the spectrum of scattered $\gamma$, $\varepsilon$ is the photopic efficiency for the energy of scattered  $\gamma$, $N$ is the number of electrons in the sample of Al, $\Delta\Omega$ is the solid angle covered by the Detector, and I/S is the number of $\Gamma$ hitting the sample per unit of surface. 
 