
\section*{Conclusion}
This report was to describe the Compton Scattering process of electromagnetic radiation. Three inorganic NaI(Tl) scintillators were used with a $^{22}$Na $\gamma$-source collimated inside a proper shaped lead box,  emitting two collinear 511~keV  and a single 1275~keV $\gamma$. The former were selected for the analysis using coincidences between the detectors.

The first part of the experiment  was developed in order to study the angular dependence of the scattered $\gamma$ energy. Fig.~\ref{Fig:Scattering_angles} shown its agreement with the theoretical values and the energy conservation of the process.

The second part, instead, was focused on the measurement of the differential cross section for a specific scattering angle. This was achieved substituting one of the three detectors, previously used as scattering center, with an Al target. Knowing the geometry of the setup and counting the events registered by the detectors, the Cross section was evaluated finding a value of $(4.3\pm0.2)\cdot 10^{-30}\ \text{m}^{2}$.  The theoretical expected value at $90^\circ$ is $\sim$1.5 $\times 10^{-30}$ m$^2$, differently to the measured one. This is mainly due to the poor statistic in the experiment and the possible errors which arise in the determination of the quantities necessary to compute the cross section.
